\begin{frame}
\frametitle{Topology}
\begin{itemize}
\item<1-> One idea in topology is to generalize the idea of continuity that you learn in your calculus class.
\item<2-> We should like to investigate a space's properties that are preserved under continuous deformation.
\item<3-> It is often said the a topologist cannot tell the difference between a coffee cup and a doughnut.
\end{itemize}
\end{frame}


\begin{frame}
\frametitle{Topology}
%\begin{center}
%\animategraphics[autoplay, loop, scale = 0.75]{4}{/CupToDoughnut/mug}{01}{58}
%\end{center}
\end{frame}

\begin{frame}
\frametitle{Topology}
\begin{itemize}
\item<1-> Topologists regard two spaces as `the same' (\textbf{homeomorphic}) whenever one space can be continuously deformed to look like the other space.
\item<2-> No `ripping' and no `gluing'.
\item<3-> In the previous example, the coffee cup and doughnut look the same to a topologist.
\item<4-> What other spaces look the same to a topologist?
\end{itemize}
\end{frame}

\begin{frame}
\frametitle{Topology}
Which of the following do topologists regard as `the same'?

\bigskip

\center{\fontsize{70}{80}\selectfont \textsf{A B C D}}

\end{frame}

\begin{frame}
\frametitle{Topology}
Which of the following do topologists regard as 'the same'?
\bigskip
\center{\fontsize{70}{80}\selectfont \textsf{ {\color{red} A} B C {\color{red} {D}}}}

\end{frame}

\begin{frame}
\frametitle{Topology}
How do we know that these are `the same'?

\center{\fontsize{70}{80}\selectfont \textsf{  A  D }}
\end{frame}

\begin{frame}
\frametitle{Topology}
Because we can find the deformation!
\frametitle{Topology}
%\begin{center}
%\animategraphics[autoplay, scale = 0.35]{2}{/AtoD/A}{1}{17}
%\end{center}
\end{frame}



\begin{frame}
\frametitle{Topology}
\begin{itemize}
\item<1-> How do we know that the following are `different'?
\end{itemize}
\center{\fontsize{70}{80}\selectfont \textsf{  A  B }}
\begin{itemize}
\item<2-> Just because we cannot find a deformation does not mean such a deformation does not exist!
\item<3-> We need more tools in order to prove that two spaces are `different'.
\item<4-> This is why algebraic topology was invented.
\end{itemize}
\end{frame}