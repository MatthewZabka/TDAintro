\begin{frame}{InteractiveJPDwB}
\begin{itemize}
\item<1-> Recall: Given a set of points $X$ and a parameter $r$, we can create a formal simplicial complex.
\item<2-> Let the $n$-simplex $\{x_0, x_1, \ldots, x_n\}$ exist if and only if $d(x_i, x_j) < r$ for all $0 \leq i,j\leq n$. 
\item<3-> We can visualize this process using InteractiveJPDwB.
\end{itemize}
\end{frame}

%-------------------------------------------------------------------------
\begin{frame}{InteractiveJPDwB}
\begin{itemize}
\item<1-> InteractiveJPDwB \cite{Wolcott2016InteractiveJPDwB} lets one visualize the generated simplicial complex for data in $\mathbb{R}^2$ and different values of $r$.
\item<2-> In other words, it demonstrates $0$-th and $1$-st degree persistence homology via barcodes.
\item<2-> Thanks to Michael Catanzaro, you can easily install this program onto your computer.
\item<3-> You can download this program from:
\begin{center}
\hyperref[https://github.com/MatthewZabka/MAA-NCS18.git]{\textcolor{blue}{\texttt{https://github.com/MatthewZabka/MAA-NCS18.git}}}
\end{center}
\end{itemize}
\end{frame}
%-----------------------------------------------------------------------------
\begin{frame}{Discuss with your groupmates!}
\begin{itemize}
\item What is the minimum number of points required so that $\beta_1 = 1$ for some value of $r$?
\item What is the minimum number of points required so that, for some $r$, we have $\beta_0 = 3$ and $\beta_1 = 2$?
\item What is the largest degree of homology that is geometrically fesible in $\mathbb{R}^2$?
\item What is the minimum number of points required to have $\beta_2 = 1$? In what dimension must the points lie?
\item What is the minimum number of points required to have $\beta_n = 1$?  In what dimension must the points lie?
\end{itemize}
\end{frame}
