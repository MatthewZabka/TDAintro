\documentclass[handout]{beamer}
\usefonttheme{serif}

\mode<presentation> {

\usetheme{Frankfurt}

%\setbeamertemplate{footline} % To remove the footer line in all slides uncomment this line
%\setbeamertemplate{footline}[page number] % To replace the footer line in all slides with a simple slide count uncomment this line

%\setbeamertemplate{navigation symbols}{} % To remove the navigation symbols from the bottom of all slides uncomment this line
}
\usepackage[T1]{fontenc}
\usepackage{tikz-cd}
\usetikzlibrary{positioning, calc}
\usetikzlibrary{matrix,arrows,decorations.pathmorphing}
\usepackage{mathtools}

\newcommand{\nul}{\mathrm{nul}}
\newcommand{\rank}{\mathrm{rank}}
\DeclareMathOperator{\coker}{coker}
\DeclareMathOperator{\Hom}{Hom}

\usepackage{graphicx} % Allows including images

\setbeamertemplate{headline}{}
\setbeamertemplate{bibliography item}{\insertbiblabel}
%----------------------------------------------------------------------------------------
%	TITLE PAGE
%----------------------------------------------------------------------------------------

\title{A Tutorial in Topological Data Analysis} % The short title appears at the bottom of every slide, the full title is only on the title page

\author{Matthew Zabka}
\institute[SMSU] % Your institution as it will appear on the bottom of every slide, may be shorthand to save space
{
 % Your institution for the title page
}
%\titlegraphic{\includegraphics[scale=0.3]{images/SMSU}}
\date{13. September 2018} % Date, can be changed to a custom date

\begin{document}
\begin{frame}
\titlepage % Print the title page as the first slide
\end{frame}

\begin{frame}
\frametitle{Overview} % Table of contents slide, comment this block out to remove it
\tableofcontents 
\end{frame}

\section{A Short Review}
%---------------------------------------------------------
\begin{frame}{A short review}
As we saw yesterday:
\begin{itemize}
\item<2-> Topology provides many descriptors about the `shape' of a space.
\item<3-> The one of these descriptors -- homology -- is computable, which makes it particularly useful in applications.
\item<4-> Given a topological space $T$ and a field $\mathbb{F}$, the $n$-th homology group $H_n(T;\mathbb{F})$ assigns 
\item<5-> In particular, the rank of the $n$-th homology group is called the $n$-th Betti number and is denoted
	\[
	\beta_n := \textrm{rank}(H_n(T;\mathbb{F})).
	\]
\end{itemize}
\end{frame}
%---------------------------------------------------------
\begin{frame}{A short review}
\begin{itemize}
\item The $n$-th Betti number tells us about the number of $n$-dimensional holes in a space.
	\begin{itemize}
	\item<2-> $\beta_0$ counts the number of connected components in a space.
	\item<3-> What is $\beta_0$ of the following space?\\
	\
	\begin{center}
	{\fontsize{50pt}{1pt}\selectfont \={I}}
	\end{center}
	\item<4-> $\beta_0 = 3$
	\end{itemize}
\end{itemize}
\end{frame}
%---------------------------------------------------------
\begin{frame}{A short review}
\begin{itemize}
\item The $n$-th Betti number tells us about the number of $n$-dimensional holes in a space.
	\begin{itemize}
	\item<1-> $\beta_1$ counts the number of `holes' in space.
	\item<2-> You can think of a hole as something through which you can stick you arm.
	\item<3-> What are $\beta_0$ and $\beta_1$ of the following space?
	\begin{center}
	{\fontsize{50pt}{12pt}\selectfont \"{B}}
	\end{center}	
	\item<4-> $\beta_0 = 3$ and $\beta_1 = 2$
	\end{itemize}
\end{itemize}
\end{frame}
%---------------------------------------------------------
\begin{frame}{A short review}
\begin{itemize}
\item The $n$-th Betti number tells us about the number of $n$-dimensional holes in a space.
	\begin{itemize}
	\item<2-> $\beta_2$ counts the number of voids.
	\item<3-> What is $\beta_2$ of the following space?
	%\begin{center}
	%\includegraphics[scale=0.5]{/home/matthew/Documents/Junk/images/S2VS2}
	%\end{center}
	\item<4-> $\beta_2 = 2$
	\item<5-> What are $\beta_0$ and $beta_1$?
	\item<6-> $\beta_0 = 1$ and $\beta_1 = 0$.
	\end{itemize}
\end{itemize}
\end{frame}

\end{document} 
